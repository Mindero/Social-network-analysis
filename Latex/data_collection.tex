Сбор информации с пользователей социальной сети не является такой простой задачей. Во"=первых, данные пользователей являются самой ценной информацией социальной сети и находятся под защитой закона. Во"=вторых, многие социалные сети используют множество динамических страниц, таких как AJAX и DHTML, для которых трудно придумать гибкую программу, способная собирать все нужные данные. И в"=третьих, становится все больше пользователей, обеспокоенных своей безопасностью, которые закрывают свою страницу от незнакомцев. 

Сбор данных зависит от трех факторов:
\begin{itemize}
    \item Выбор начальных узлов
    \item Алгоритм выбора узлов
    \item Размер графа
\end{itemize}
Правильный выбор начальных узлов позволяет избежать данных низкого качества. Как правило, выбирают тех пользователей, которые обладают наибольшим количеством связей с другими\cite{wong2014design}.

Алгоритм выбора решает, какой узел выбрать следующим. Группа жадных алгоритмов и поиск в ширину являются самыми популярными алгоритмами выбора\cite{CrawlingONS}. Для обхода сайтов в сети Интернет, одноранговых сетей и других огромных графов используется случайное блуждание на графе, то есть следующая вершина выбирается случайно из всех соседей предыдущей вершины\cite{gjoka2011practical}.

Размер графа же определяет, когда нужно прекратить собирать данные.