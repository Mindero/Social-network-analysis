Анализ социальных сетей используются в разных сферах.
\subsection{Образование}
Из сети, вершинами которой являются школьники или студенты, а ребра "--- в каких отношениях обучающиеся, можно получить много полезной информации. Так, анализ такой сети позволяет выявить, насколько школьник склонен начать курить\cite{mercken2012longitudinal}. Кроме того, изучение таких сетей позволяет понять влияние подростковой агрессии\cite{sijtsema2010forms}, успеваемость в учёбе, привычки и характер.

\subsection{Медицина}
Анализ социальных сетей используется в эпидемиологии человека в качестве инструмента для изучения потенциальной передачи патогенов, таких как ВИЧ, туберкулез, гепатит В и сифилис. В профилактической ветеринарной медицине же этот подход дает преимущества для изучения характера и степени контактов между животными или фермами, что в конечном итоге приводит к
лучшему пониманию потенциального риска распространения заболевания в восприимчивой к нему
популяции\cite{martinez2009social}.


\subsection{Политика}
Анализ графа, полученного путём разбора текста политиков, выделения фраз и глаголов и установления связей между ними, позволяет определить положение партий, основные темы, которые они затрагивают, а также отношения внутри партий\cite{politics}. 

\subsection{Терроризм}
Применение анализа социальных сетей для выявления и предотвращения терроризма стали применять после трагичных событиях $11$ сентября $2001$ года. Ученые, анализируя сети, исследовали психологические и социологические тенденции для того, чтобы расшифровать профиль членов террористических групп и понять их мотивы, а также для объяснения стратегических и тактических решений террористических организаций\cite{perliger2011social}.
Благодаря анализу социальных сетей удалось обнаружить сеть, имеющая $4$ кластера, общее количество террористов которой доходило до $172$\cite{sageman2004understanding}.

 
\subsection{Маркетинг}
Сегодня всё больше компании используют вирусный маркетинг, который использует социальные сети для повышения осведомленности бренда. Главным распространителем информации являются сами получатели информации, так как человек, получающий сообщение, исходящее от лица незаинтересованного, например, от знакомого, охотнее запомнит бренд. Для поиска таких лиц используется поиск лидеров мнений, то есть применяется степень близости\cite{closeness_marketing} в социальном графе.
  
Анализ социальных сетей также применяют в рекомендательной системе, таким образом маркетологи cоздают более эффективные маркетинговые кампании.









