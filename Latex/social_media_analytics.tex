\subsection{Метрики}
\textit{Метрики} "--- это числовые показатели, которые отображают характеристику объектов, сегментов, групп и их связей. Так как анализ социальных сетей чаще всего применяется в изучении человеческих взаимоотношений и взаимосвязей, то многие метрики пришли из социологии.

\textit{Гомофилия} "--- это степень, с которой индивид сформирует связь с схожим участником. Схожесть может быть определена по половому признаку, расе, возрасту и т.д. Эту концепцию лучше всего описывает пословица <<Рыбак рыбака видит из далека>>.

\textit{Соседство} "--- это склонность к появлению связей между индивидами по географическому признаку.

\textit{Обоюдность} или \textit{взаимность} "--- это степень, с которой у двух участников образуется двусторонняя связь\cite{kadushin2012understanding}.

\textit{Закрытость сети} "--- метрика полноты сети. Напрмер, закрытость сети говорит о том, сколько ваших друзей являются друзьями\cite{wong2014design}.

\textit{Мостом} называют связь, которая обеспечивает единственный путь между двумя вершинами. Например, мост между $A$ и $B$ означает, что путь информации от любого контака $A$ до любого контакта $B$ содержит эту связь\cite{Granovetter}.

\textit{Кликой} в теории графов определяют как подграф, в котором каждая пара вершин соединена ребром. В социологии же кликой называют группу людей, которые взаимодействуют друг с другом чаще и интенсивнее\cite{wong2014design}.

\subsection{Центральность}
Центральность это ещё одна метрика социального графа. Она определяет, насколько  <<важной>> является вершина в сети, основываясь на некоторых объективных критериях\cite{hansen2010analyzing}. Понятие <<важности>> узла можно определять по"=разному, поэтому существует несколько видов центральности, каждая из которых применима в определенной ситуации. Выделим несколько из них.

Одной из самых простых трактовок <<важности>> вершины является её степень, то есть количество входящих и выходящих из этой вершины ребер. Вычисление степеней каждой вершины в сети и сортировка их в порядке убывания даст ранг локальной (степенной) центральности. Эта центральность называется локальной, поскольку она выделяет точки, которые хорошо связаны между собой в непосредственной близости\cite{scott2012social}.

Можно подойти к вопросу <<важности>> вершины с другой стороны. Выделим вершины, которые ближе всего ко всем остальным. Степень близости узла вычисляется по формуле
\begin{equation*}
    C(v) = \frac{N}{\sum_u{d(v,u)}},
\end{equation*}
где $d(v,u)$ равно расстоянию между вершина $v$ и $u$, а $N$ "--- количество узлов. Близость применятеся в разных сферах. Так, в библиометрии степень близость показывает насколько в среднем близка одна дисциплина ко всем другим в сети\cite{bibliometrics}. В социальных сетях же установлено, что продвижение товара пользователям с наибольшей степенью близости ведёт к огромной выгоде\cite{closeness_marketing}.

Также существует другой вариант вычисления степени близости в неориентированном графе:
\begin{equation*}
    H(v) = \frac{1}{N - 1} \cdot  \sum_{u\vert u \neq v}{\frac{1}{d(u,v)}},
\end{equation*}
где $1/d(u,v) = 0$, если нет пути из $v$ в $u$, $N$ "--- количество узлов. Центральность, основанная на такой формуле, называется гармонической.

Кроме того, определение <<важной>> вершины можно дать через кратчайший путь. Степень посредничества вершины $v$ равна числу кратчайших путей, проходящих через эту вершину. Формально это записывается таким образом:
\begin{equation*}
    C_B(v) = \sum_{s \neq u \neq t }{\frac{\sigma_{st}(v) }{\sigma_{st}}},
\end{equation*}
где $\sigma_{st}$ "--- это количество кратчайших путей из $s$ в $t$, а $\sigma_{st}(v)$ количество этих путей, проходящих через $v$. В телекоммуникационной сети, узел с наивысшей степенью посредничества имеет большой контроль над сетью, так как через нее проходит больше всего информации.
