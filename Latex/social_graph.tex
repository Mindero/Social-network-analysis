Под социальной сетью понимается множество пользователей, которые могут выступать во взаимодействие друг с другом. С формальной точки зрения такие сети удобно представлять в виде графов и применять для анализа математические модели.

Впервые термин был введен в 1954 году социологом из <<Манчестерской школы>> Джеймсом Барнсом в работе <<Классы и собрания в норвежском островном приходе>>. В своей работе он охарактеризовал социальную сеть следующим образом: <<Каждый человек имеет определенный круг друзей, а эти друзья имеют, в свою очередь собственных друзей. Некоторые из друзей одного человека знают друг друга, а другие "--- нет. Я нашел удобным говорить о такого рода полях как о сетях. Под этим мне видится система точек, некоторые из которых соединины между собой. Точками этой системы являются люди а линии соединения этих точек указывают, какие люди взаимодействуют друг с другом>>.\cite{Sazanov}

Бурное развитие социальный граф получил после конференции Facebook F8 в 2007 году, на котором Марк Цукенберг представил программное обеспечение, которое собирает данные пользователей и их взаимоотношения.\cite{CBSNews}

Анализ социальных сетей, таких как Facebook, Twitter, Vk и т.д., имеет огромную ценность для бизнеса, так как позволяет понять ценность продукта для конкретной аудитории и как стоит его продвигать. Кроме того, анализ социальных сетей позволяет выдавать персонализированную рекламу, что имеет коммерческую выгоду для компаний. Социальными сетями пользуются опасные преступники, террористы для осуществление деструктивной деятельности. Анализ социальных сетей позволяет выявлять подобные группы и защищать обычных людей от их влияния.\cite{hansen2010analyzing}