\documentclass[bachelor, och, referat, times]{SCWorks}
% параметр - тип обучения - одно из значений:
%    spec     - специальность
%    bachelor - бакалавриат (по умолчанию)
%    master   - магистратура
% параметр - форма обучения - одно из значений:
%    och   - очное (по умолчанию)
%    zaoch - заочное
% параметр - тип работы - одно из значений:
%    referat    - реферат
%    coursework - курсовая работа (по умолчанию)
%    diploma    - дипломная работа
%    pract      - отчет по практике
%    pract      - отчет о научно-исследовательской работе
%    autoref    - автореферат выпускной работы
%    assignment - задание на выпускную квалификационную работу
%    review     - отзыв руководителя
%    critique   - рецензия на выпускную работу
% параметр - включение шрифта
%    times    - включение шрифта Times New Roman (если установлен)
%               по умолчанию выключен
\usepackage[T2A]{fontenc}
\usepackage[utf8]{inputenc}
\usepackage{graphicx}

\usepackage[sort,compress]{cite}
\usepackage{amsmath}
\usepackage{amssymb}
\usepackage{amsthm}
\usepackage{fancyvrb}
\usepackage{longtable}
\usepackage{array}
\usepackage[english,russian]{babel}
\usepackage{minted}
% Используется автором репозитория
%\usemintedstyle{xcode}
% Этот пакет включает в себя аналогичный Times New Roman шрифт.
% Необходим для успешной компиляции для UNIX-систем ввиду отсутствия TNR в нем.
% Можно использовать и для Windows.
%\usepackage{tempora}


\usepackage[colorlinks=false]{hyperref}


\newcommand{\eqdef}{\stackrel {\rm def}{=}}

\newtheorem{lem}{Лемма}

% % При использовании biblatex вместо bibtex
%\usepackage[style=gost-numeric]{biblatex}
%\addbibresource{thesis.bib}

\begin{document}

% Кафедра (в родительном падеже)
\chair{информатики и программирования}

% Тема работы
\title{Анализ социальных сетей}

% Курс
\course{1}

% Группа
\group{151}

% Факультет (в родительном падеже) (по умолчанию "факультета КНиИТ")
%\department{факультета КНиИТ}

% Специальность/направление код - наименование
%\napravlenie{02.03.02 "--- Фундаментальная информатика и информационные технологии}
%\napravlenie{02.03.01 "--- Математическое обеспечение и администрирование информационных систем}
%\napravlenie{09.03.01 "--- Информатика и вычислительная техника}
\napravlenie{09.03.04 "--- Программная инженерия}
%\napravlenie{10.05.01 "--- Компьютерная безопасность}

% Для студентки. Для работы студента следующая команда не нужна.
%\studenttitle{Студентки}

% Фамилия, имя, отчество в родительном падеже
\author{Янченко Вадима Александровича}

% Заведующий кафедрой
%\chtitle{доцент, к.\,ф.-м.\,н.} % степень, звание
%\chname{С.\,В.\,Миронов}

%Научный руководитель (для реферата преподаватель проверяющий работу)
\satitle{доцент, к.\,ф.-м.\,н.} %должность, степень, звание
\saname{А.\,П.\,Грецова}

% Руководитель практики от организации (только для практики,
% для остальных типов работ не используется)
\patitle{к.\,ф.-м.\,н., доцент}
\paname{А.\,П.\,Грецова}

% Семестр (только для практики, для остальных
% типов работ не используется)
\term{2}

% Наименование практики (только для практики, для остальных
% типов работ не используется)
\practtype{учебная}

% Продолжительность практики (количество недель) (только для практики,
% для остальных типов работ не используется)
\duration{2}

% Даты начала и окончания практики (только для практики, для остальных
% типов работ не используется)
\practStart{01.07.2016}
\practFinish{14.07.2016}

% Год выполнения отчета
\date{2023}

\maketitle

% Включение нумерации рисунков, формул и таблиц по разделам
% (по умолчанию - нумерация сквозная)
% (допускается оба вида нумерации)
%\secNumbering


\tableofcontents

% Раздел "Обозначения и сокращения". Может отсутствовать в работе
% \abbreviations
% \begin{description}
%     \item ... "--- ...
%     \item ... "--- ...
% \end{description}

% Раздел "Определения". Может отсутствовать в работе
%\definitions

% Раздел "Определения, обозначения и сокращения". Может отсутствовать в работе.
% Если присутствует, то заменяет собой разделы "Обозначения и сокращения" и "Определения"
%\defabbr


% Раздел "Введение"

\intro
Сегодня каждый из нас использует социальные сети. Уже стало привычкой проверить утром публикации любимых сообществ, <<лайкнуть>> новое фото приятеля, переслать смешные картинки друзьям. Миллиарды людей ежедневно создают триллионы подобной информации. Социальные сети хранят все действия пользователей, а также многие социально"=демографические данные этих пользователей в открытом доступе. 

Современные программные средства позволяют собрать, проанализировать, визуализировать и получить вывод из собранных данных, которые сформированны миллионами сообщений, ссылок, постов, фотографий, и видео. Таким образом обнаруживаются связи между объектами, которые невозможно было бы получить без анализа социальных сетей. Изучение таких данных в определенных темах является одним из способов анализа тенденций изменения общественного
мнения в широком спектре вопросов, а результаты
анализа могут быть использованы в различных областях для решения задач практической направленности, включая задачи антитеррористической
направленности, прогнозирование потребности,
политические прогнозы, маркетинговые исследования, оценка репутационных рисков компании
или физического лица\cite{Tomsk_research}.

Цель данной работы "--- изучение основных методов анализа социальных сетей и применение их на практике.
% Целью данной работы является сбор и анализ данных подписчиков сообщества <<СГУ факультета КНиИТ>> \footnote{\url{https://vk.com/sgu_kniit}} и сообщества <<Душевно>>\footnote{\url{https://vk.com/somethinggoodforyoursoul}}.
% В ходе работы должны быть решены следующие задания:
% \begin{enumerate}
%     \item Изучение понятий социального графа
%     \item Изучение методов анализа социальных сетей, 
%     \item Сбор данных пользователей, подписанных на выбранные сообщества
%     \item Анализ полученного графа
% \end{enumerate}

% После введения — серии \section, \subsection и т.д.
\section{Социальный сеть}
Под социальной сетью понимается множество пользователей, которые могут выступать во взаимодействие друг с другом. С формальной точки зрения такие сети удобно представлять в виде графов и применять для анализа математические модели.

Впервые термин был введен в 1954 году социологом из <<Манчестерской школы>> Джеймсом Барнсом в работе <<Классы и собрания в норвежском островном приходе>>. В своей работе он охарактеризовал социальную сеть следующим образом: <<Каждый человек имеет определенный круг друзей, а эти друзья имеют, в свою очередь собственных друзей. Некоторые из друзей одного человека знают друг друга, а другие "--- нет. Я нашел удобным говорить о такого рода полях как о сетях. Под этим мне видится система точек, некоторые из которых соединины между собой. Точками этой системы являются люди а линии соединения этих точек указывают, какие люди взаимодействуют друг с другом>>.\cite{Sazanov}

Бурное развитие социальный граф получил после конференции Facebook F8 в 2007 году, на котором Марк Цукенберг представил программное обеспечение, которое собирает данные пользователей и их взаимоотношения.\cite{CBSNews}

Анализ социальных сетей, таких как Facebook, Twitter, Vk и т.д., имеет огромную ценность для бизнеса, так как позволяет понять ценность продукта для конкретной аудитории и как стоит его продвигать. Кроме того, анализ социальных сетей позволяет выдавать персонализированную рекламу, что имеет коммерческую выгоду для компаний. Социальными сетями пользуются опасные преступники, террористы для осуществление деструктивной деятельности. Анализ социальных сетей позволяет выявлять подобные группы и защищать обычных людей от их влияния.\cite{hansen2010analyzing}


% Раздел "Заключение"
\conclusion

%Библиографический список, составленный вручную, без использования BibTeX
%
%\begin{thebibliography}{99}
%  \bibitem{Ione} Источник 1.
%  \bibitem{Itwo} Источник 2
%\end{thebibliography}

%Библиографический список, составленный с помощью BibTeX
%
\nocite{*}
\bibliographystyle{gost780uv}
\bibliography{thesis}

% % При использовании biblatex вместо bibtex
%\printbibliography

% Окончание основного документа и начало приложений
% Каждая последующая секция документа будет являться приложением
\appendix

\end{document}