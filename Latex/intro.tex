Сегодня каждый из нас использует социальные сети. Уже стало привычкой проверить утром публикации любимых сообществ, <<лайкнуть>> новое фото приятеля, переслать смешные картинки друзьям. Миллиарды людей ежедневно создают триллионы подобной информации. Социальные сети хранят все действия пользователей, а также многие социально"=демографические данные этих пользователей в открытом доступе. 

Современные программные средства позволяют собрать, проанализировать, визуализировать и получить вывод из собранных данных, которые сформированны миллионами сообщений, ссылок, постов, фотографий, и видео. Таким образом обнаруживаются связи между объектами, которые невозможно было бы получить без анализа социальных сетей. Изучение таких данных в определенных темах является одним из способов анализа тенденций изменения общественного
мнения в широком спектре вопросов, а результаты
анализа могут быть использованы в различных областях для решения задач практической направленности, включая задачи антитеррористической
направленности, прогнозирование потребности,
политические прогнозы, маркетинговые исследования, оценка репутационных рисков компании
или физического лица\cite{Tomsk_research}.

Цель данной работы "--- изучение основных методов анализа социальных сетей и применение их на практике.
% Целью данной работы является сбор и анализ данных подписчиков сообщества <<СГУ факультета КНиИТ>> \footnote{\url{https://vk.com/sgu_kniit}} и сообщества <<Душевно>>\footnote{\url{https://vk.com/somethinggoodforyoursoul}}.
% В ходе работы должны быть решены следующие задания:
% \begin{enumerate}
%     \item Изучение понятий социального графа
%     \item Изучение методов анализа социальных сетей, 
%     \item Сбор данных пользователей, подписанных на выбранные сообщества
%     \item Анализ полученного графа
% \end{enumerate}