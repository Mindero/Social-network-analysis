Сегодня каждый из нас использует социальные сети. Уже стало привычкой проверить утром публикации своих любимых сообществ, лайкнуть новое фото своего приятеля, переслать смешные картинки своим друзьям. Миллиарды людей ежедневно создают триллионы подобной информации. Социальные сети хранят все действия пользователей, а также многие социально-демографические данные этих пользователей в открытом доступе. 

Современные программные средства позволяют собрать, проанализировать, визуализировать и получить вывод из собранных данных, которые сформированны миллионами сообщений, ссылок, постов, фотографиями, и видео. Таким образом обнаруживаются связи между вещями, которые невозможн было бы обнаружить без анализа социальных сетей. Изучение таких данных в контексте определенных тем является одним из способов анализа тенденций изменения общественного
мнения в широком спектре вопросов, а результаты
анализа могут быть использованы в различных областях для решения задач практической направленности, включая задачи антитеррористической
направленности, прогнозирование потребности,
политические прогнозы, маркетинговые исследования, оценка репутационных рисков компании
или физического лица.\cite{Tomksk_researh}

Целью данной работы является сбор и анализ данных подписчиков сообщества <<СГУ факультета КНиИТ>> \footnote{\url{https://vk.com/sgu_kniit}} и сообщества <<Душевно>>\footnote{\url{https://vk.com/somethinggoodforyoursoul}} на выявление лидеров мнений.

%Мы живем в эпоху сетей. 